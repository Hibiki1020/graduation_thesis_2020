\chapter*{概要}
\addcontentsline{toc}{chapter}{概要}{季節や天候,時間などによる環境変化に対してロバストなビジュアルローカリゼーションの達成はロボティクスにおいて重要な目標である.従来手法では機械学習を点群地図の生成と自己位置推定の両方に取り入れることでこれを達成してきた.しかし,二点間の間隔の大きい点群地図は視界外にある点が尤度算出の妨げになる,1つの点が持つ情報の重要度が上がりセンサーの誤差に弱くなるなど,尤度算出上の問題を抱えていた. \par 本研究ではこれらの問題解決のためにメッシュ地図を導入することで入り組んだ地形に対して頑強な尤度算出を可能とするビジュアルローカリゼーションを提案する.検証では点群地図とメッシュ地図での自己位置推定を行い,算出される尤度を比較することで本手法の有効性を示す.}
