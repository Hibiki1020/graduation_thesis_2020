\chapter{結論}
提案手法により, 点群地図特有の問題点であるスパース性によって入り組んだ環境において尤度算出が困難になる問題と少ない数の点で尤度算出を行うことによってセンサ誤差やセグメンテーションの誤分類に弱くなる問題が解決された. また, 提案手法が先行研究と比べて自己位置推定において良好な結果を示したことにより本手法の有用性を示すことができた. \par 最後に, 提案手法の今後の展望について述べる. 提案手法はPCL等の既存のライブラリを活用することで自己位置推定を行っていたが, ライブラリによる処理の中にはメッシュの影を表現するための計算など自己位置推定には必要のない処理が数多くあり, これらが計算時間の増大につながった. ロボットや自動運転において使用できる計算リソースは限られているため提案手法を実環境下で使用する場合, 自己位置推定のための最適化が必須となる. これらの問題点を解決できたとき, 提案手法は自動運転だけでなく屋内環境で動作するロボットなど幅広い分野で活躍することが期待される.